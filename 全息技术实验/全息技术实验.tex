\documentclass{ctexart}
\usepackage{geometry}
\usepackage{fancyhdr}
\usepackage{graphicx}
\usepackage{booktabs}
\usepackage{amsmath}
\usepackage{tikz}
\usepackage{array}
\usepackage{zhnumber} % change section number to chinese
\renewcommand\thesection{\zhnum{section}}
\renewcommand \thesubsection {\arabic{subsection}}
\CTEXsetup[format={\Large\bfseries}]{section}

\geometry{
    a4paper,
    left=3.18cm,
    right=3.18cm,
    top=2.54cm,
    bottom=2.54cm
}

\pagestyle{fancy}
\fancyhf{}
\renewcommand{\headrulewidth}{0.7pt} % 设置页眉横线粗细
\fancyhead[L]{\kaishu\large 大学物理实验报告} % 在左侧设置页眉文字
\fancyhead[R]{\kaishu\large 哈尔滨工业大学(深圳) } % 在右侧设置页眉文字
\fancyfoot[R]{\raisebox{1\baselineskip}{\thepage}} % 将页数放在右下角


\setlength\headwidth{\textwidth}

\begin{document}

\noindent
\begin{center}
\textbf{
\begin{tabular}{p{2.4cm}p{2.4cm}p{4cm}p{4cm}}
    班级 \hrulefill & 学号 \hrulefill & 姓名 \hrulefill & 教师签字 \hrulefill \\
\end{tabular}
\begin{tabular}{p{6cm}p{3.6cm}p{3.6cm}}
    实验日期 \hrulefill & 预习成绩 \hrulefill & 总成绩 \hrulefill
\end{tabular}
{\noindent}	 \rule[-10pt]{\textwidth}{0.7pt}
}\end{center}

\begin{center}
    \Large \textbf{实验内容 \underline{全息技术实验}}
\end{center}

\section{预习内容}
简述全息照相的纪录与再现原理。
\newpage

~\\
~\\
~\\
~\\
~\\
~\\
~\\
~\\
~\\
~\\
~\\
~\\
~\\
~\\
~\\
~\\

\section{原始数据记录}

\begin{table}[h]
    \renewcommand\arraystretch{1.6}
    \centering
    \caption{光路信息}
    \label{tab:D}
    \begin{tabular}{|m{2cm}<{\centering}|m{2cm}<{\centering}|m{2.3cm}<{\centering}|m{2.5cm}<{\centering}|m{3.5cm}<{\centering}|}
        \hline
        物光光强 & 参考光光强 & 物光光程(dm) & 参考光光程(dm) & 参考光与物光的夹角(°)\\
        \hline
         &  & & &   \\
        \hline
    \end{tabular}
\end{table}

\begin{tikzpicture}[remember picture,overlay]
    \node[anchor=south east,inner sep=100pt] at (current page.south east) {
        \renewcommand{\arraystretch}{1.5} % 表格行高倍数
        \setlength{\tabcolsep}{18pt}    
    \begin{tabular}{|c|c|}
        \hline
        \LARGE  教师 & \LARGE  姓名 \\
        \hline
        \LARGE \kaishu 签字 &  \\
        \hline
        \end{tabular}
    };
\end{tikzpicture}
\newpage
\section{实验现象分析及结论}
\textbf{试分析哪些因素会对全息成像有影响。}

光源的相干性,光路的稳定性,物光与参考光夹角,环境的温度、湿度,成像距离等等都对实验有一定影响。

\section{讨论题}
\subsection{试比较全息照相与普通照相的异同点。}
\subsubsection{工作原理:}

\begin{itemize}
    

    \item 普通照相:普通照相使用光学透镜将场景的光线聚焦到感光材料(例如胶卷或传感器)上,记录下被聚焦光线反射的图像。
    \item 全息照相:全息照相则是利用激光光源产生的相干光束,将场景的全部信息记录在光敏介质上,形成一个记录了物体形状、大小和位置等三维信息的全息图。
\end{itemize}    

\subsubsection{成像方式:}
\begin{itemize}  
    \item 普通照相:普通照相只能记录场景的表面信息,即物体在二维平面上的投影。
    \item 全息照相:全息照相记录了场景的完整三维信息,包括物体的深度、形状和位置等。
\end{itemize}
\subsection{为什么用白光照射全息照片会出现彩带?为什么说观察到彩带即说明拍摄成功?}
这是因为白光包含了多种波长的光,而全息照片中记录了被照射物体的全部信息,包括了各种波长的光的干涉图样。当白光照射到全息照片上时,不同波长的光在干涉时产生的干涉条纹会反射或透射出来,形成了彩色的干涉条纹,也就是彩带。

观察到彩带通常被认为是全息照片拍摄成功的标志,这是因为彩带的出现表明了全息照片中记录了大量的细微信息,并且在还原出三维场景时,这些信息将对图像的清晰度和深度有所帮助。彩带的存在暗示着全息图像中存在丰富的相干光信息,这意味着照片具有更高的分辨率和更好的深度感。
\subsection{参考光与物光之间夹角的大小对成像有何影响?}
\begin{itemize}
    \item 较小夹角下拍摄的全息图像能够更准确地捕捉到物体的细微深度变化,提供更加清晰和真实的三维立体效果。
    \item 较大夹角下拍摄的全息图像能够捕捉到更广阔的场景范围,但可能会牺牲一定的深度信息分辨率。
    \item 在较小夹角下,物光和参考光的相位差变化较小,因此对震动和环境扰动的容忍度较高,全息图像更加稳定。
    \item 较大的夹角会增加全息图像中干涉条纹的密度,但同时也增加了光学系统对相干性的要求。
\end{itemize}
\end{document}