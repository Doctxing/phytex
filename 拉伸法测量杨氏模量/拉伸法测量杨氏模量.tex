\documentclass{ctexart}
\usepackage{geometry}
\usepackage{fancyhdr}
\usepackage{graphicx}
\usepackage{booktabs}
\usepackage{amsmath}
\usepackage{diagbox}
\usepackage{tikz}
\usepackage{array}
\xeCJKsetup{CJKmath=true} 
\usepackage{zhnumber} % change section number to chinese
\renewcommand\thesection{\zhnum{section}}
\renewcommand \thesubsection {\arabic{subsection}}
\CTEXsetup[format={\Large\bfseries}]{section}

\geometry{
    a4paper,
    left=3.18cm,
    right=3.18cm,
    top=3.04cm,
    bottom=3.04cm
}

\pagestyle{fancy}
\fancyhf{}
\renewcommand{\headrulewidth}{0.7pt} % 设置页眉横线粗细
\fancyhead[L]{\kaishu\large 大学物理实验报告} % 在左侧设置页眉文字
\fancyhead[R]{\kaishu\large 哈尔滨工业大学(深圳) } % 在右侧设置页眉文字
\fancyfoot[R]{\thepage} % 将页数放在右下角


\setlength\headwidth{\textwidth}

\begin{document}

\noindent
\begin{center}
\textbf{
\begin{tabular}{p{2.4cm}p{2.4cm}p{4cm}p{4cm}}
    班级 \hrulefill & 学号 \hrulefill & 姓名 \hrulefill & 教师签字 \hrulefill \\
\end{tabular}
\begin{tabular}{p{6cm}p{3.6cm}p{3.6cm}}
    实验日期 \hrulefill & 预习成绩 \hrulefill & 总成绩 \hrulefill
\end{tabular}
{\noindent}	 \rule[-10pt]{\textwidth}{0.7pt}
}\end{center}

\begin{center}
    \Large \textbf{实验内容 \underline{拉伸法测量杨氏模量}}
\end{center}

\section{实验目的}

~\\
~\\

\section{实验预习}

\subsection{杨氏模量的物理意义是什么?国标单位是什么?}

~\\
~\\
~\\

\subsection{光杠杆法的原理是什么,是如何实现微小量放大的?(画出测量原理光路图)。}

~\\
~\\
~\\
~\\
~\\
~\\
~\\
~\\
~\\
~\\

\subsection{本实验需要测量哪些物理量来间接得到杨氏模量?}

\newpage
\section{实验现象及数据记录}

\begin{table}[h]
    %\renewcommand\arraystretch{1.6}
    \centering
    \caption{一次性数据}
    \label{tab:once_data}
    \begin{tabular}{|m{2cm}<{\centering}|m{2cm}<{\centering}|m{2cm}<{\centering}|}
        \hline
        $L (mm)$&$H (mm)$&$D (mm)$ \\
        \hline
         & & \\
        \hline
    \end{tabular}
\end{table}

\begin{table}[h]
    %\renewcommand\arraystretch{1.6}
    \centering
    \caption{金属丝直径测量数据	螺旋测微器零差$d_0=\ \ \ \ \ \ \ \ \ mm$}
    \label{tab:D}
    \begin{tabular}{|m{2cm}<{\centering}|m{1.3cm}<{\centering}|m{1.3cm}<{\centering}|m{1.3cm}<{\centering}|m{1.3cm}<{\centering}|m{1.3cm}<{\centering}|m{1.3cm}<{\centering}|m{1.3cm}<{\centering}|}
        \hline
        序号$i$ & 1 & 2 & 3 & 4 & 5 & 6 & 平均值 \\
        \hline
        直径视值$d_{视i} \ $(mm)&  & & & & & &  \\
        \hline
    \end{tabular}
\end{table}

\begin{table}[h]
    %\renewcommand\arraystretch{1.6}
    \centering
    \caption{加减力时标尺刻度与对应拉力数据}
    \label{tab:datas}
    \begin{tabular}{|m{3.5cm}<{\centering}|m{0.6cm}<{\centering}|m{0.6cm}<{\centering}|m{0.6cm}<{\centering}|m{0.6cm}<{\centering}|m{0.6cm}<{\centering}|m{0.6cm}<{\centering}|m{0.6cm}<{\centering}|m{0.6cm}<{\centering}|m{0.6cm}<{\centering}|m{0.6cm}<{\centering}|}
        \hline
        序号$i$&1&2&3&4&5&6&7&8&9&10 \\
        \hline
        拉力视值$f_i$ \ \ \ \ \ \ \ \ (kg) & & & & & & & & & & \\
        \hline
        加力时标尺刻度$x_i^+$ (mm) & & & & & & & & & & \\
        \hline
        减力时标尺刻度$x_i^-$ (mm) & & & & & & & & & & \\
        \hline
        平均标尺刻度$xi=\frac{(x_i^+ + x_i^−)}{2}$ (mm) & & & & & & & & & & \\
        \hline
        标尺刻度改变量$\Delta x_i = x_{i+5}−x_i$ (mm) & & & & & & \multicolumn{5}{m{0.6cm}<{\centering}|}{} \\
        \hline
    \end{tabular}
\end{table}

\begin{tikzpicture}[remember picture,overlay]
    \node[anchor=south east,inner sep=100pt] at (current page.south east) {
        \renewcommand{\arraystretch}{1.5} % 表格行高倍数
        \setlength{\tabcolsep}{18pt}    
    \begin{tabular}{|c|c|}
        \hline
        \LARGE  教师 & \LARGE  姓名 \\
        \hline
        \LARGE \kaishu 签字 &  \\
        \hline
        \end{tabular}
    };
\end{tikzpicture}

\newpage

\section{数据处理}

(要有详细的计算过程,推导不确定度的表达式,计算杨氏模量及其不确定度,给出完整的测量结果表达形式)

金属丝的平均值:

$$ \overline{d} = \overline{d_视} - d_0 =  0.546 + 0.055 = 0.601 mm $$

根据实验原始数据,拉力视值每增加1kg,标尺刻度的改变量平均值:

$$ \overline{\Delta x}_{by1} = \frac{\sum_{i=1}^{5}\Delta x_i}{5 \cdot 5} = 3.588 mm $$

所以金属丝的平均伸长量:

$$ \overline{\Delta L} = \frac{D}{2H} \overline{\Delta x}_{by1} = \frac{44.22 mm}{2 \times 684.0 mm}\times 3.588 mm = 0.1160 mm $$

根据杨氏模量的表达式:

$$ E = \frac{4 \Delta F}{\pi d^2} \cdot \frac{L}{\Delta L} = \frac{4\Delta m g}{\pi d^2} \cdot \frac{L}{\Delta L}$$

得杨氏模量的计算值为:

$$ \overline{E} = \frac{4\times 1.00 kg \times 9.8 N/kg}{\pi \times (0.601 mm)^2} \times \frac{726.5 mm}{0.1160 mm} = 2.16354 \times 10^{11} N/m^2 $$


~\\
接下来计算不确定度:

根据杨氏模量的表达式$E = \frac{8\Delta mgLH}{\pi Dd^2}\cdot\frac{1}{\overline{\Delta x}_{by1}}$,得到合成不确定度表达式

$$ E_E = \sqrt{\frac{U_L^2}{L^2}+\frac{U_H^2}{H^2}+\frac{U_D^2}{D^2}+\frac{U_{\overline{\Delta m}}^2}{{\overline{\Delta m}}^2}+\frac{4 U_{\overline{d}}^2}{{\overline{d}}^2}+\frac{U_{\overline{\Delta x}}^2}{\overline{\Delta x}^2}} $$

1. 公式中 $L, H, D, \Delta 𝑚$ 只有B类不确定度,有 $ U = \frac{\Delta_仪}{\sqrt{3}} $

\begin{table}[h]
    %\renewcommand\arraystretch{1.6}
    \centering
    \caption{误差限表}
    \label{tab:delta_yi}
    \begin{tabular}{|m{2cm}<{\centering}|m{2cm}<{\centering}|m{2cm}<{\centering}|}
        \hline
        \textbf{\heiti 量具名称}&\textbf{\heiti 测量参数}&\textbf{\heiti 误差限} $\Delta_仪$ \\
        \hline
        钢卷尺&$L,H$&$0.8 mm $ \\
        \hline
        游标卡尺&$D$&$0.02 mm $ \\
        \hline
        螺旋测微器&$d$&$0.004 mm $ \\
        \hline
        数字拉力计&$m$&$0.005 kg $ \\
        \hline
        标尺&$\Delta x$&$0.5 mm$ \\
        \hline
    \end{tabular}
\end{table}

2. 而d(使用螺旋测微器测量)既有A类不确定度,也有B类不确定度,故有:

$$ S_{\overline{d}} = \sqrt{\frac{\sum_{i=1}^{6}\left(d_视-\overline{d_视}\right)^2}{6\times(6-1)}} = 6.0553 \times 10^{-4} mm $$

则计算得:

$$ U_d = \sqrt{(S_{\overline{d}})^2+(\frac{\Delta_仪}{\sqrt{3}})^2} = 2.38747 \times 10^{-3} mm $$


3. 而$\Delta x$为间接测量量,要计算其合成不确定度$U_{\Delta x}$,同样的我们有:

$$ \overline{\Delta x} = \frac{\sum_{i=1}^{5}\Delta x_i}{5} = 17.49 mm $$

有:

$$ S_{\overline{\Delta x}} = \sqrt{\frac{\sum_{i=1}^{5}\left(\Delta x -\overline{\Delta x}\right)^2}{5\times(5-1)}} = 0.58189 $$

应注意,这里$\Delta x$也是间接量,此处B类不确定度应当乘上一个系数,

$$u_B = \sqrt{2} \frac{\Delta_仪}{\sqrt{3}} = 0.40825 $$

合成得到:
$$ U_{\Delta x}=\sqrt{(S_{\overline{\Delta x}})^2+(u_B)^2} = 0.71082$$


则最终的合成不确定度有:

$$ E_E = \left[ \left(\frac{0.8}{\sqrt{3} \times 726.5 }\right)^2+\left(\frac{0.8}{\sqrt{3} \times 684.0}\right)^2+\left(\frac{0.02}{\sqrt{3} \times 44.22}\right)^2+\right. $$
$$\left.\left(\frac{0.005}{\sqrt{3} \times 1.00}\right)^2+4  \left(\frac{2.38747 \times 10^{-3}}{0.601}\right)^2+\left(\frac{0.71082}{17.49}\right)^2\right]^{\frac{1}{2}} = 4.052\% $$

~
$$U_E = \overline{E} \cdot E_E = 0.088 \times 10^{11} N/m^2 $$
$$ E = \overline{E} \pm U_E = (2.164 \pm 0.088) \times 10^{11} N/m^2 $$
$$P = 68.3\% $$
\newpage

\section{实验结论及误差分析}
结论:测量得到金属丝的杨氏模量为 $(2.164±0.088)×10^{11} N/m^2 $,不确定度 $ E_E =4.052\% $,置信概率为 $68.3\% $,查询资料得到铁的杨氏模量为$2.11 \times 10^{11} N/m^2$,结果符合实际值。 

从计算不确定度的步骤可以发现,所有不确定度中数值最大的是$\frac{U_{\overline{\Delta x}}^2}{\overline{\Delta x}^2}$项,因此标尺是最主要的误差来源。 
\section{讨论问题}

\subsection{材料相同,但粗细、长度不同的两根钢丝,它们的杨氏模量是否相同?}
杨氏模量只与材料有关,所以这两根钢丝的杨氏模量相同。
\subsection{从误差分析的角度分析为什么同是长度测量,需要采用不同的量具?}
不同测量工具的的量程和误差大小不同。如果测量的长度较长,就必须选择量程大且误差大的量具;如果测量的长度较小,就应该选择量程小且误差小的量具。
\subsection{实验过程中为什么加力和减力过程,施力螺母不能回旋?}
由于实验器材的原因,如果在加力和减力过程将施力螺母回旋,则会产生回程误差,降
低实验结果的准确性。
\subsection{用逐差法处理数据的优点是什么?应该注意什么问题?}
优点:可以利用到的每一组实验数据,减小误差。 

注意:所测量的数据应是偶数(4, 6, 8, …)组。
\end{document}