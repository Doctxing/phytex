\documentclass{ctexart}
\usepackage{geometry}
\usepackage{fancyhdr}
\usepackage{graphicx}
\usepackage{booktabs}
\usepackage{amsmath}
\usepackage{tikz}
\usepackage{array}
\xeCJKsetup{CJKmath=true} 
\usepackage{zhnumber} % change section number to chinese
\renewcommand\thesection{\zhnum{section}}
\renewcommand \thesubsection {\arabic{subsection}}
\CTEXsetup[format={\Large\bfseries}]{section}

\geometry{
    a4paper,
    left=3.18cm,
    right=3.18cm,
    top=3.04cm,
    bottom=3.04cm
}

\pagestyle{fancy}
\fancyhf{}
\renewcommand{\headrulewidth}{0.7pt} % 设置页眉横线粗细
\fancyhead[L]{\kaishu\large 大学物理实验报告} % 在左侧设置页眉文字
\fancyhead[R]{\kaishu\large 哈尔滨工业大学(深圳) } % 在右侧设置页眉文字
\fancyfoot[R]{\thepage} % 将页数放在右下角


\setlength\headwidth{\textwidth}

\begin{document}

\noindent
\begin{center}
\textbf{
\begin{tabular}{p{2.4cm}p{2.4cm}p{4cm}p{4cm}}
    班级 \hrulefill & 学号 \hrulefill & 姓名 \hrulefill & 教师签字 \hrulefill \\
\end{tabular}
\begin{tabular}{p{6cm}p{3.6cm}p{3.6cm}}
    实验日期 \hrulefill & 预习成绩 \hrulefill & 总成绩 \hrulefill
\end{tabular}
{\noindent}	 \rule[-10pt]{\textwidth}{0.7pt}
}\end{center}

\begin{center}
    \Large \textbf{实验内容 \underline{液体表面张力系数测量}}
\end{center}

\section{实验预习}
\subsection{什么是表面张力? 液体表面张力系数与哪些因素有关?}
\subsection{拉脱法测量液体表面张力的实验原理是什么?}

\newpage
\section{实验现象及原始数据记录}
\subsection{吊环的内、外直径(单位:mm)}

\begin{table}[h]
    \renewcommand\arraystretch{1.6}
    \centering
    \caption{吊环的测量}
    \label{tab:m}
    \begin{tabular}{|m{1.8cm}<{\centering}|m{1.5cm}<{\centering}|m{1.5cm}<{\centering}|m{1.5cm}<{\centering}|m{1.5cm}<{\centering}|m{1.5cm}<{\centering}|m{1.5cm}<{\centering}|}
        \hline
        测量次数& 1 & 2 & 3 & 4 & 5 & 平均值 \\
        \hline
        内径$D_i$ & & & & & & \\
        \hline
        外径$D_o$ & & & & & & \\
        \hline
    \end{tabular}
\end{table}

~

\subsection{利用逐差法求仪器的转换系数$K$:}
~\\
先记录砝码盘等作为初始读数$V_0 = \rule{1.3cm}{0.4pt} mV $ ,然后每次增加一个砝码500mg,(该标准砝码符合国家标准,相对误差为0.05\%)
~\\

\begin{table}[h]
    \renewcommand\arraystretch{1.3}
    \centering
    \caption{吊环的测量}
    \label{tab:k}
    \begin{tabular}{|m{2cm}<{\centering}|m{2cm}<{\centering}|m{2cm}<{\centering}|m{3.5cm}<{\centering}|}
        \hline
        砝码质量 $10^{-6} Kg $ & 增重读数 $V_1' (mV)$ & 减重读数 $V_1'' (mV)$ & $V_i = \frac{V_i{'}+V_i{''}}{2} (mV)$ \\
        \hline
        0.00 & & & \\
        \hline
        500.00 & & & \\
        \hline
        1000.00 & & & \\
        \hline
        1500.00 & & & \\
        \hline
        2000.00 & & & \\
        \hline
        2500.00 & & & \\
        \hline
        3000.00 & & & \\
        \hline
        3500.00 & & & \\
        \hline
    \end{tabular}
\end{table}

~

利用逐差法求出每 500 mg 对应的电子秤的读数 $\Delta V$ ,则 $\overline{K} = \frac{mg}{\Delta V} = \rule{1.5cm}{0.4pt} N/mV $ 
\newpage
\subsection{用拉脱法求拉力对应的电子秤读数:}

\begin{table}[h]
    \renewcommand\arraystretch{1.6}
    \centering
    \caption{室温下表面张力系数测量表}
    \label{tab:T1}
    \begin{tabular}{|m{1.8cm}<{\centering}|m{3cm}<{\centering}|m{2.5cm}<{\centering}|m{4cm}<{\centering}|}
        \multicolumn{4}{c}{水温(室温)$\rule{1.5cm}{0.4pt} ^\circ C$,  电子秤初始读数$V_0 = \rule{1.5cm}{0.4pt} mV$} \\
        \hline
        测量次数& 拉脱时最大读数 $V_1 (mV)$ & 吊环读数 $V_2 (mV)$ & 表面张力对应读数(mV) $V_i^\ast = V_1-V_2$\\
        \hline
        1 & & & \\
        \hline
        2 & & & \\
        \hline
        3 & & & \\
        \hline
        4 & & & \\
        \hline
        5 & & & \\
        \hline
        平均值 & & & $\overline{V} = \rule{2.5cm}{0.4pt} $\\
        \hline
    \end{tabular}
\end{table}

\begin{table}[!h]
    \renewcommand\arraystretch{1.6}
    \centering
    \caption{不同温度下表面张力系数测量表}
    \label{tab:T2}
    \begin{tabular}{|m{1.8cm}<{\centering}|m{3cm}<{\centering}|m{2.5cm}<{\centering}|m{4cm}<{\centering}|}
        \multicolumn{4}{c}{水温(室温)$\rule{1.5cm}{0.4pt} ^\circ C$,  电子秤初始读数$V_0 = \rule{1.5cm}{0.4pt} mV$} \\
        \hline
        测量次数& 拉脱时最大读数 $V_1 (mV)$ & 吊环读数 $V_2 (mV)$ & 表面张力对应读数(mV) $V_i^\ast = V_1-V_2$\\
        \hline
        1 & & & \\
        \hline
        2 & & & \\
        \hline
        3 & & & \\
        \hline
        4 & & & \\
        \hline
        5 & & & \\
        \hline
        平均值 & & & $\overline{V} = \rule{2.5cm}{0.4pt} $\\
        \hline
    \end{tabular}
\end{table}

\begin{tikzpicture}[remember picture,overlay]
    \node[anchor=south east,inner sep=100pt] at (current page.south east) {
        \renewcommand{\arraystretch}{1.5} % 表格行高倍数
        \setlength{\tabcolsep}{18pt}    
    \begin{tabular}{|c|c|}
        \hline
        \LARGE  教师 & \LARGE  姓名 \\
        \hline
        \LARGE \kaishu 签字 &  \\
        \hline
        \end{tabular}
    };
\end{tikzpicture}

\newpage

\section{数据处理}
\subsection{测量室温下水的表面张力系数,并计算不确定度}
据实验信息可知:

\begin{align*}
    \overline{L} &= \pi \left(\overline{D_{\text{内}}} + \overline{D_{\text{外}}}\right) \\
    \overline{\alpha} &= \frac{\overline{K} \cdot \overline{V}}{\overline{L}} \\
    \left(\frac{\Delta \alpha}{\overline{\alpha}}\right)^2 &= \left(\frac{\Delta K}{K}\right)^2 + \left(\frac{\Delta V}{V}\right)^2 + \left(\frac{\Delta L}{L}\right)^2 \\
    \alpha &= \overline{\alpha} \pm \Delta \alpha 
\end{align*}

首先计算 $500mg$ 对应的电子秤的读数,为了表示方便,记 $ \delta V_{i,i+4} = \frac{V_{i+4}-V_{i}}{4} $

$$ \Delta V = \frac{\sum\limits_{i = 1}^{4} \delta V_{i,i+4} }{4} = 536.9 \mu V $$
计算逐差法对应的不确定度,在单次对电压的测量中,误差满足正态分布规律,C取3
$$ U_s = \frac{0.001mV}{3} $$


可以求得$V_i$对应的不确定度以及$\delta V_{i,i+4}$对应的不确定度:

\begin{align*}
    U_{V_i} &= \sqrt{2 \times \frac{U_s^2}{2^2}} = 235.7 nV \\
    U_{\delta V_{i,i+4}} &= \sqrt{2 \times \frac{U_{V_i}^2}{4^2}} = 83.3 nV
\end{align*}

将A类不确定度和B类不确定度合成有:

$$ U_{V} = \sqrt{U_{\delta V_{i,i+4}}^2 + \frac{\sum\limits_{i=1}^{4} \left(\delta V_{i,i+4} - \Delta V\right)^2}{4 \times (4-1)}} = 110.24 nV $$

然后计算转换系数:

$$ \overline{K} = \frac{mg}{\Delta V} = 9.127 \times 10^{-3} N/mV $$

计算转换系数的不确定度 

$$ U_K = \left|U_V\frac{\partial K}{\partial \Delta V}\right| = \frac{mg U_V}{\Delta V^2} = 1.874 \times 10^{-6} N/mV $$
接下来计算室温下的水表面张力系数。 


首先计算表面张力对应的测量值读数:

$$ \overline{V^\ast} = \frac{\sum\limits_{i=1}^{5} V_i^\ast}{5} = 1.657 mV  $$


接下来计算单次得到表面张力对应系数的不确定度: 

$$ U_{V_i^\ast} = \sqrt{2 \times \frac{U_s^2}{2^2}} = 235.7 nV$$

将A类不确定度和B类不确定度合成有:

$$ U_{\overline{V^\ast}} = \sqrt{U_{V_i^\ast}^2 + \frac{\sum\limits_{i=1}^{5} \left(V_i^\ast - \overline{V^\ast} \right)^2}{5 \times (5-1)}} = 2.719 \mu V $$
接下来求吊环内外径的不确定度,

先求得均值:

\begin{align*}
    \overline{D_i} &= \frac{\sum\limits_{j=1}^{5} D_{ij}}{5} = 32.78 mm \\
    \overline{D_o} &= \frac{\sum\limits_{j=1}^{5} D_{oj}}{5} = 34.75 mm
\end{align*}

求取测量吊环内外径的不确定度,游标卡尺极限误差0.02mm,误差服从均匀分布:

\begin{align*}
    A_i = \sqrt{\frac{\sum\limits_{j=1}^{5}\left(D_{ji}-\overline{D_i}\right)^2}{5 \times (5-1)}} = 7.75 \mu m \\
    A_o = \sqrt{\frac{\sum\limits_{j=1}^{5}\left(D_{jo}-\overline{D_o}\right)^2}{5 \times (5-1)}} = 5.00\mu m \\
\end{align*}

又有其B类不确定度$U_r = \frac{0.02}{3} mm $,有

\begin{align*}
    U_{\overline{D_i}} &= \sqrt{U_r^2 + A_i^2} = 10.22 \mu m \\
    U_{\overline{D_o}} &= \sqrt{U_r^2 + A_o^2} = 8.33 \mu m 
\end{align*}
现求取表面张力系数:

$$ \alpha = \frac{\overline{K} \cdot \overline{V}}{\pi\left(\overline{D_i}+\overline{D_o}\right)} = 7.13 \times 10^{-2} N/m $$

应用合成不确定度计算方法化简得到:

$$ E = \sqrt{\frac{U_K^2}{\overline{K}^2} + \frac{U_{\overline{V^\ast}}^2}{\overline{V^\ast}^2} + \frac{U_{\overline{D_i}}^2 + U_{\overline{D_o}}^2}{\left(\overline{D_i}+\overline{D_o}\right)^2}} = 0.2\% $$

求出不确定度:

$$ U = E \alpha = 0.000164 N/m $$
$$ \alpha = (7.13 \pm 0.02) \times 10^{-2} N/m $$
$$ P = 68.3 \% $$



\subsection{从附录中查出室温下水的表面张力系数$\alpha$的理论值,把实验结果与此值比较求相对误差,并进行分析。}

经查表得到在$25.3^\circ C$下其表面张力系数为$7.18 N/m$,计算相对误差:

$$ \sigma = \frac{7.18-7.13}{7.13} = 0.7\% $$

此误差不大,可以接受。误差可能来自吊环底面与水面不平行,同时也可能有来自拉断液膜时读数不及时造成的影响。

\subsection{测量不同温度下表面张力系数,并与室温下水的表面张力系数理论值作分析比较。}

计算过程与上述计算过程相同,省略计算过程。计算结果得到:

$$ E = \sqrt{\frac{U_K^2}{\overline{K}^2} + \frac{U_{\overline{V^\ast}}^2}{\overline{V^\ast}^2} + \frac{U_{\overline{D_i}}^2 + U_{\overline{D_o}}^2}{\left(\overline{D_i}+\overline{D_o}\right)^2}} = 1.2\% $$
$$ U = E \alpha = 0.000856 N/m $$
$$ \alpha = (7.02 \pm 0.08) \times 10^{-2} N/m $$
$$ P = 68.3 \% $$

经查表得到在$30.9^\circ C$下其表面张力系数为$7.11 N/m$,计算相对误差:

$$ \sigma = \frac{7.11-7.02}{7.02} = 1.3\% $$

经计算得到$30.9^\circ C$下其表面张力系数比$25.3^\circ C$要小一些,这是因为温度升高,分子热运动加剧,液体分子之间的吸引力会减小,于是表面张力系数减小。

\section{实验结论及现象分析}
实验结果表明,测量值较理论值偏小,可能是因为拉出吊环后吊环上沾有少量水份,但是在实验误差允许范围内接受实验结果。

测量仪器转换系数时,需要等到砝码盘稳定后再读数,否则由于加速度的影响会出现误差。主要的误差来自于吊环底面与水面不平行。判断是否平行,可以在降低水面的过程中观察水面是否在吊环外侧的同一纹理上。调整水位时不能太剧烈,否则水中产生明显的水波会干扰拉力计的施力。调整水位过快可能会导致拉断液面时测力计还未来得及探测到峰值,因此要缓慢调整液面高度。

\section{讨论题}
\subsection{在推导液体表面张力系数测量公式中作了哪些近似?式中各量的物理意义是什么?}

\begin{itemize}
    \item 将液膜边缘的切线方向近似为重力加速度方向(湿润角 $ \to 0 $)。
    \item 认为拉起的液膜质量很小进而忽略这部分的重力(mg)计算。
\end{itemize}

$$\alpha = \frac{F-m_0g}{\pi (D_i+D_o)}$$

本式中,$F$为拉起吊环的拉力,$m_0g$为吊环的重力。$D_i, D_o$为吊环的内、外径。

\subsection{若考虑拉起液膜的重量,实验结果应如何修正?}
考虑拉起液膜的重量,计算公式应改为:

$$\alpha = \frac{F-(m+m_0)g}{\pi (D_i+D_o)}$$

若能求出液膜的质量,则可以通过吊环的质量按比例折算到计算公式中对实验结果进行
修正。 
\end{document}