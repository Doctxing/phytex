\documentclass{ctexart}
\usepackage{geometry}
\usepackage{fancyhdr}
\usepackage{graphicx}
\usepackage{booktabs}
\usepackage{amsmath}
\usepackage{tikz}
\usepackage{array}
\usepackage{zhnumber} % change section number to chinese
\renewcommand\thesection{\zhnum{section}}
\renewcommand \thesubsection {\arabic{subsection}}
\CTEXsetup[format={\Large\bfseries}]{section}

\geometry{
    a4paper,
    left=3.18cm,
    right=3.18cm,
    top=2.54cm,
    bottom=2.54cm
}

\pagestyle{fancy}
\fancyhf{}
\renewcommand{\headrulewidth}{0.7pt} % 设置页眉横线粗细
\fancyhead[L]{\kaishu\large 大学物理实验报告} % 在左侧设置页眉文字
\fancyhead[R]{\kaishu\large 哈尔滨工业大学(深圳) } % 在右侧设置页眉文字
\fancyfoot[R]{\raisebox{1\baselineskip}{\thepage}} % 将页数放在右下角


\setlength\headwidth{\textwidth}

\begin{document}

\noindent
\begin{center}
\textbf{
\begin{tabular}{p{2.4cm}p{2.4cm}p{4cm}p{4cm}}
    班级 \hrulefill & 学号 \hrulefill & 姓名 \hrulefill & 教师签字 \hrulefill \\
\end{tabular}
\begin{tabular}{p{6cm}p{3.6cm}p{3.6cm}}
    实验日期 \hrulefill & 预习成绩 \hrulefill & 总成绩 \hrulefill
\end{tabular}
{\noindent}	 \rule[-10pt]{\textwidth}{0.7pt}
}\end{center}

\begin{center}
    \Large \textbf{实验内容 \underline{示波器实验(虚拟仿真)}}
\end{center}

\section{实验预习}

\subsection{示波器的基本结构主要有哪些?}
~ \\
~ \\
~ \\
~ \\ 
~ \\
~ \\
~ \\
~ \\
~ \\
~ \\ 
~ \\
~ \\
\subsection{李萨如图形形成原理是什么?如何利用李萨如图形测量待测信号频率?}

\newpage

\section{实验现象及原始数据记录}

实验模式

\subsection{测量示波器自备方波输出信号的周期(时基分别为0.1、0.2、0.5 $ms/DIV$)。}

\begin{table}[h]
    %\renewcommand\arraystretch{1.6}
    \centering
    \caption{方波信号频率测量}
    \label{tab:self}
    \begin{tabular}{|m{3cm}<{\centering}|m{2cm}<{\centering}|m{2cm}<{\centering}|m{2cm}<{\centering}|}
        \hline
        选择时基(ms)&0.1&0.2&0.5\\
        \hline
        方波信号(Hz)& & & \\
        \hline
    \end{tabular}
\end{table}

\subsection{用示波器测量信号发生器输出的方波信号频率。}

\begin{table}[h]
    %\renewcommand\arraystretch{1.6}
    \centering
    \caption{信号发生器输出的方波信号频率测量}
    \label{tab:self_1}
    \begin{tabular}{|m{2cm}<{\centering}|c|c|c|c|c|c|c|c|c|c|}
        \hline
        时基(ms)&0.5&0.5&0.2&0.2& 0.1& 0.1 & 0.1 & 0.1 & 0.1 & 0.05 \\
        \hline
        格数 & & & & & & & & & & \\
        \hline
        周期(ms) & & & & & & & & & & \\
        \hline
        频率(Hz)& & & & & & & & & & \\
        \hline
    \end{tabular}
\end{table}

\subsection{三角波信号的测量。}

\subsubsection{选择信号发生器输出三角波,频率分别为500、1K、1.5K、2K Hz;}
\subsubsection{测量各个频率下三角波的上升时间、下降时间和周期。}

\begin{table}[h]
    %\renewcommand\arraystretch{1.6}
    \centering
    \caption{不同频率下三角波信号测量}
    \label{tab:self_2}
    \begin{tabular}{|m{3.5cm}<{\centering}|m{2cm}<{\centering}|m{2cm}<{\centering}|m{2cm}<{\centering}|m{2cm}<{\centering}|m{2cm}<{\centering}|}
        \hline
        频率(Hz) & 500 & 1000 & 1500 & 2000 \\
        \hline
        三角波信号上升时间(ms) & & & & \\
        \hline
        三角波信号下降时间(ms)& & & & \\
        \hline
        三角波信号周期(ms)& & & & \\
        \hline
    \end{tabular}
\end{table}

\subsection{观察李萨如图形并测频率。}
用待测信号源(正弦信号)和信号发生器分别接 y 轴(CH2通道)和 x 轴(CH1通道),取$\frac{f_x}{f_y}$为1、 1/2、2时,测出对应的$f_x$和$f_y$,记录有关图形并求出待测信号的频率。

\newpage
\begin{table}[h]
    %\renewcommand\arraystretch{1.6}
    \centering
    \caption{利用李萨如图形测量信号频率}
    \label{tab:self_3}
    \begin{tabular}{|m{2cm}<{\centering}|m{3cm}<{\centering}|m{3cm}<{\centering}|m{3cm}<{\centering}|}
        \hline
        $f_x / f_y$ & 1 & 2/1 & 1/2 \\
        \hline
        待测信号频率(Hz) & & & \\
        \hline
        信号发生器频率(Hz)& & & \\
        \hline
        各$\frac{f_x}{f_y}$比例对应的李萨如图像  & & & \\
        \hline
    \end{tabular}
\end{table}

\begin{tikzpicture}[remember picture,overlay]
    \node[anchor=south east,inner sep=100pt] at (current page.south east) {
        \renewcommand{\arraystretch}{1.5} % 表格行高倍数
        \setlength{\tabcolsep}{18pt}    
    \begin{tabular}{|c|c|}
        \hline
        \LARGE  教师 & \LARGE  姓名 \\
        \hline
        \LARGE \kaishu 签字 &  \\
        \hline
        \end{tabular}
    };
\end{tikzpicture}

\newpage
\subsection{实验结论及现象分析}

\begin{itemize}
    \item 在测量示波器自身方波输出信号时,将示波器探头插入后,调节时基和电压,均得到了方波图形,并对所示图形进行了读数。
    \item 将示波器探头摘下,将信号发生器调至预期频率及波形,并接入示波器,不断调整方波信号频率以及示波器的时基,读数并进行记录。
    \item 将波形调至三角波,示波器上出现三角波形,读取上升下降所用时间。
    \item 调节信号发生器为正弦波形,并连接待测原件,调至x-y示波,并不断调节信号发生器频率,使得示波器上出现稳定的预期李萨如图像,并记录此时的示波器频率。
\end{itemize}

经过实验,了解到示波器的基本使用操作,以及一些基础的测量方法,能够利用李萨如图来计算待测曲线的频率。

\subsection{讨论题}

\textbf{假定在示波器的y轴输入一个正弦电压,所用的水平扫描频率为120Hz,在荧光屏上出现三个稳定的正弦波形,那么输入信号的频率是多少?这是否是测量信号频率的好方法?为什么?}

为了找到输入信号的频率,可以利用水平扫描频率和观察到的波形数量之间的关系。由于在荧光屏上观察到三个稳定的正弦波形,因此可以推断输入信号的频率是水平扫描频率的三倍,即:

输入信号的频率 $=3×120Hz=360Hz$

这种方法可能并不是测量信号频率的最佳方法,因为它依赖于示波器的水平扫描频率和观察到的波形数量之间存在整数倍的关系。这种方法对于特定情况下的频率测量可能是有效的,但在其他情况下可能会产生误导性的结果。

测量信号频率的更好方法是直接使用示波器的频率测量功能来测量输入信号的频率。现代示波器通常具有内置的频率测量功能,可以准确地测量输入信号的频率,并且不受水平扫描频率或波形数量的限制。因此,直接使用示波器的频率测量功能是更可靠和准确的方法。

\end{document}